Pour conclure, L'AFC de cette base de données nous a permis de mettre en valeur différentes corrélations.
Notamment entre la tranche d'âge et le régime alimentaire, le temps consacré aux études et le temps de sommeil ou le temps de sommeil et le régime alimentaire.
Ces corrélations nous ont permis de mettre en évidence certains comportements chez des groupes d'étudiants comme par exemple le fait que les étudiants les plus âgés ont tendance à avoir une alimentation modérément saine. 
Nous avons aussi montré que les résultats pouvaient différer lorsque nous séparons les individus dépressifs et non dépressifs.
Cela nous a permis de mettre en évidence des phénomènes s'apparentant au paradoxe de Simpson.
En plus de cela nous avons pu remarquer des différences de tendance entre les étudiants dépressifs et non dépressifs mais aussi des similarités, ce qui peut permettre de mieux comprendre les conséquences de cette maladies sur le rythme de vie des étudiants.

