Pour conclure, L'AFC de cette base de données nous a permis de mettre en valeur différentes corrélations.
Notamment entre la tranche d'age et le régime alimentaire, le temps consacré aux études et le temps de sommeil ou le temps de sommeil et le régime alimentaire.
Nous avons aussi montré que les résultats pouvaient différer lorsque nous séparons les individus dépressifs et non dépressifs. Nous avons notamment mis en évidence des phénomènes s'apparentant au paradoxe de Simpson.
