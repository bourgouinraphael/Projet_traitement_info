\subsection{Etude du lien entre âge et habitudes alimentaires}

\subsubsection{Préparation des données}

Nous avons ensuite voulu etudier le lien entre l'âge et les habitudes alimentaires.
Cepandant, la valeur du champ âge est directement l'âge, ce qui représente trop de catégories (une vingtaine) par rapport au nombre d'individus présent dans le jeu de données ($500$).
Nous avons donc décidé de répartir les individus en tranches d'âges: les $18-22$, $22-26$, $26-30$ et $30+$.
Nous pouvons maintenant voir dans le tableau de contingence en figure \ref{tab:contTableAgeDietary} que le nombre d'individu est suffisament élevé dans chaque catégorie pour pouvoir faire une AFC ayant du sens.

\begin{figure}[!h]
\begin{center}
  \begin{tabular}{|c|c|c|c|}
    \hline 
    & \multicolumn{3}{|c|}{Dietary Habits}\\ 
    \hline
    Age & Healthy & Moderate & Unhealthy \\ 
    \hline 
    18-22 & 35 & 39 & 42 \\ 
    \hline 
    22-26 & 36 & 34 & 43 \\ 
    \hline 
    26-30 & 41 & 34 & 48 \\ 
    \hline 
    30+ & 49 & 65 & 36 \\ 
    \hline
  \end{tabular}
\end{center}
\caption{Table de contingence entre les habitudes alimentaires et les tranches d'âges}
\label{tab:contTableAgeDietary}
\end{figure}

\subsubsection{Test Chi-deux}

Après exécution du test Chi-deux, la p-valeur obtenue est de $\approx6\%$ ce qui et au dessus de la p-valeur usuellement utilisée pour ce test, mais n'est pas non plus très éloigné.
Ainsi, nous avons quand même décidé de poursuivre l'analyse car cette p-valeur semble tout de même indiquer au moins une faible corrélation entre les deux variables.

\subsubsection{Valeurs propres et cercle des corrélations}

Après exécution de l'AFC, les deux composantes principales obtenues expliquent $100\%$ de la variance, avec la première en expliquant $\approx 97\%$.
Ainsi, la quasi totalité des corrélations seront montrées par la première composante, soit l'axe des abscisses du cercle des corrélations donné en figure \ref{fig:corrAgeDiretary}.

\begin{figure}[!h]
  \begin{center}
    \includegraphics[width=0.55\textwidth]{Images/Age_Dietary_all/Corr_circle.png}
  \end{center}
  \caption{Cercle des corrélations de l'AFC sur les tranches d'âges et les habitudes alimentaires}
  \label{fig:corrAgeDiretary}
\end{figure}

