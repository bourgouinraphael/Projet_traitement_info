Le présent rapport étudie des données publiques portant sur la dépression chez les étudiants \citep{dataset}. Afin de traiter et analyser ces données, nous avons choisis l'Analyse Factorielle des Correspondances (abrégé en AFC), les données étant categorielles. Notre objectif est d'identifier des relations entre différentes catégories, selon si l'on regarde les individus dépréssifs, les individus non dépréssifs ou tous les individus. Ainsi, on pourra potentiellement constater certains effets de la dépréssion à travers les différences de corrélations entre dépréssifs et non dépréssifs.

Nous commencerons par présenter le jeu de données que nous utilisons en section \ref{datasetSection}.
Ensuite nous présenterons la méthode utilisée dans ce rapport, l'AFC, en section \ref{AFCSection}. Finalement, nous analyserons les résultats obtenus dans la section \ref{AnalysisSection} avant de conclure en section \ref{conclusionSection}.
