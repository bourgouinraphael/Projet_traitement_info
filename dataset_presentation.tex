Nous avons choisi d'étudier le jeu de données \textit{Depression Student Dataset} \citep{dataset}, constitué de données académiques (pression académique par exemple), économiques (stress financier par exemple), ainsi que sur des habitudes de vie (temps de sommeil par exemple) recueillies sur 502 individus, la moité étant dépréssifs. 

Nous avons utilisé le language \textit{Python} \citep{python} sur un \textit{jupyter notebook} \citep{jupyter} pour l'analyse . De plus, nous avons utilisé les librairies \textit{pandas} \citep{pandas} afin d'importer et gérer les données, \textit{matplotlib} \citep{matplotlib} pour tracer les graphiques, \textit{prince} \citep{prince} afin de réaliser l'AFC ainsi que \textit{scipy} \citep{scipy}, qui nous a permis de réaliser les tests chi-deux.

Le code écrit dans le cadre de notre projet est trouvable sur le dépôt github du projet \citep{repo}.
