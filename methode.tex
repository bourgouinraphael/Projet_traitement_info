L'Analyse Factorielle de Correspondance est une technique permettant d'analyser des données qualitatives. 
Plus précisément, elle permet d'analyser les relations entre 2 variables qualitatives catégorielles.
Une représentation commune de ces données est la table de contingence, pour laquelle un exemple est fourni dans la figure \ref{cont-table}. 

\begin{figure}[!h]
\begin{center}
  \begin{tabular}{| c | c | c | c | c | c |}
    \hline
    & \multicolumn{5}{|c|}{Sleep duration}\\
    \hline
    Study satisafaction & Less than 5 hours & 5-6 hours & 7-8 hours & More than 8 hours & Total\\
    \hline
    1.0 & 23 & 19 & 20 & 24 & 86 \\ 
    \hline 
    2.0 & 25 & 25 & 25 & 25 & 100\\ 
    \hline 
    3.0 & 19 & 25 & 33 & 26 & 103\\ 
    \hline 
    4.0 & 29 & 29 & 31 & 27 & 116\\ 
    \hline 
    5.0 & 27 & 25 & 19 & 26 & 97\\ 
    \hline
    Total & 123 & 123 & 128 & 128 & 502 \\ 
    \hline

  \end{tabular}
\end{center}
  \caption{Exemple de table de contingence}
  \label{cont-table}
\end{figure}

\subsection{Notations}

Dans la suite, on notera $n$ le nombre total d'instances, $V_1$ la première variable (de taille $I$), $V_2$ la seconde (de taille $J$) et $x_{ij}$ le nombre d'instances étant dans la catégorie $i$ de la variable $V_1$ et dans la catégorie $j$ de la variable $V_2$.
On définit alors $X = (x_{ij})_{1 \leq i \leq I, 1 \leq j \leq J}$ la table de contingence. On peut alors parler des valeurs marginales des lignes et colonnes, dont les formules sont:

\begin{equation}
  x_{i\bullet} = \sum_{j=1}^J x_{ij} \qquad 
  x_{\bullet j} = \sum_{i=1}^I x_{ij}.
\end{equation}

Cependant, on préfèrera toujours travailler sur la table des probabilités définie par $f_{ij} = \frac{x_{ij}}{n}$, pour lesquelles on peut aussi définir les valeurs marginales avec: 
\begin{equation}
  f_{i\bullet} = \sum_{j=1}^J f_{ij} \qquad
  f_{\bullet j} = \sum_{i=1}^I f_{ij}.
\end{equation}

\subsection{Test d'indépendance}

On souhaite dans un premier temps vérifier si les variables $V_1$ et $V_2$ sont indépendantes, ce qui est le cas si $\forall i, j, f_{ij} \approx f_{i,\bullet}f_{\bullet, j}$, on définit alors $\hat{f}_{ij} = f_{i,\bullet}f_{\bullet, j}$ la probabilité théorique sous l'hypothèse d'indépendance des variables. 
De manière similaire, on définit $\hat{x}_{ij} = n\hat{f}_{ij}$ les données théoriques sous l'hypothèse d'indépendance. 

On peut alors procéder au test d'indépendance $\chi^2$, qui consiste à:
\begin{itemize}
  \item Calculer la distance $\chi_{obs}^2 = \sum_{(i,j)} \frac{(x_{ij} - \hat{x}_{ij})^2}{\hat{x}_{ij}}$ 
  \item Fixer une $p$-value (usuellement à $0.05$)
  \item Calculer le degré de liberté $df = (I - 1)(J - 1)$ 
  \item Déterminer $\chi_{critical}^2$ ou une $p$-valeur à l'aide d'une table 
  \item Si $\chi_{obs}^2 \leq \chi_{critical}^2$ ou $p$-valeur $\geq 5\%$ alors les variables sont indépendantes. Sinon elles sont corrélées
\end{itemize}

\subsection{Diagonalisation}

Maintenant que l'on a confirmé que les variables sont corrélées, nous pouvons utiliser une technique plus précise afin d'obtenir plus d'informations, à savoir ici la diagonalisation de la matrice des probabilités $F = (f_{ij})$.

On suppose ici qu'étudier $F$ revient à étudier $\tilde{F} = D_IFD_J$ avec $D_I = diag(\frac{1}{\sqrt{f_{1 \bullet}}}, \dots, \frac{1}{\sqrt{f_{I\bullet}}})$ et $D_J = diag(\frac{1}{\sqrt{f_{\bullet 1}}}, \dots, \frac{1}{\sqrt{f_{\bullet J}}})$.
On réalisera 2 diagonalisations, une sur $\tilde{F}\tilde{F}^T$ pour étudier les lignes et une sur $\tilde{F}^T\tilde{F}$ pour étudier les colonnes.
On représente ensuite chaque catégorie de $V_1$ et $V_2$ dans un cercle des corrélations à la manière de l'ACP.
Une propriété importante ici est le fait que les valeurs propres issues des 2 diagonalisations sont identiques, ce qui va nous permettre de représenter sur le même cercle les 2 variables. Dans ce graphique, si une catégorie de $V_1$ et de $V_2$ est proche tout en étant éloignée de l'origine, cela montrera une corrélation entre ces deux catégories.

On pourra aussi calculer la contribution de chaque ligne/colonne dans les composantes principales afin de mieux interpréter les résultats.
