L'Analyse Factorielle de Correspondance est une technique permettant d'analyser des données qualitatives. 
Plus précisément, elle permet d'analyser les relations entre 2 variables qualitatives. Un représentation commune de ces données est la table de contingence, pour lequel un exemple est fourni dans la figure \ref{cont-table}. 

\begin{figure}[!h]
\begin{center}
  \begin{tabular}{| c | c | c | c | c | c |}
    \hline
    & \multicolumn{5}{|c|}{Sleep duration}\\
    \hline
    Study satisafaction & Less than 5 hours & 5-6 hours & 7-8 hours & More than 8 hours & Total\\
    \hline
    1.0 & 23 & 19 & 20 & 24 & 86 \\ 
    \hline 
    2.0 & 25 & 25 & 25 & 25 & 100\\ 
    \hline 
    3.0 & 19 & 25 & 33 & 26 & 103\\ 
    \hline 
    4.0 & 29 & 29 & 31 & 27 & 116\\ 
    \hline 
    5.0 & 27 & 25 & 19 & 26 & 97\\ 
    \hline
    Total & 123 & 123 & 128 & 128 & 502 \\ 
    \hline

  \end{tabular}
\end{center}
  \caption{Exemple de table de contingence}
  \label{cont-table}
\end{figure}

\subsection{Notation}

Dans la suite, on notera $n$ le nombre total d'instances, $V_1$ la première variable (de taille $I$), $V_2$ la seconde (de taille $J$) et $x_{ij}$ le nombre d'instances étant dans la catégorie $i$ de la variable $V_1$ et dans la catégorie $j$ de la variable $V_2$.
On définit alors $X = (x_{ij})_{1 \leq i \leq I, 1 \leq j \leq J}$ la table de contingence. On peut alors parler des valeurs marginales des lignes et colonnes, dont les formules sont:
\[
  x_{i,\bullet} = \sum_{j=1}^J x_{ij} 
\]
\[
  x_{\bullet, j} = \sum_{i=1}^I x_{ij}
\]

Cepandant, on préfèrera toujours travailler sur la table des probabilités définie par $f_{ij} = \frac{x_{ij}}{n}$, pour lesquelles on peut aussi définir les valeurs marginales avec: 
\[
  f_{i,\bullet} = \sum_{j=1}^J f_{ij}
\]
\[
  f_{\bullet, j} = \sum_{i=1}^I f_{ij}
\]

\subsection{Test d'indépendance}
